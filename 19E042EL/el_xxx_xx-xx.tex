\documentclass[10pt, a4paper, twoside]{article}

\usepackage[utf8]{inputenc}
\usepackage[T1, T2A]{fontenc}
\usepackage[]{cmsrb}
\usepackage[margin=.5in, top = 0.75in, bottom=0.75in]
{geometry}

\usepackage{lastpage}

\usepackage{ifthen}

\usepackage[serbian]{babel}

%\setcounter{errorcontextlines}{999}
%\documentclass{article}
%\usepackage[T1]{fontenc}
%\usepackage[latin1]{inputenc}
\usepackage{soulutf8}
\usepackage{color}
\makeatletter
\newcommand*{\whiten}[1]{\llap{\textcolor{white}{{\the\SOUL@token}}\hspace{#1pt}}}
\DeclareRobustCommand*\myul{%
    \def\SOUL@everyspace{\underline{\space}\kern\z@}%
    \def\SOUL@everytoken{%
     \setbox0=\hbox{\the\SOUL@token}%
     \ifdim\dp0>\z@
        \raisebox{\dp0}{\underline{\phantom{\the\SOUL@token}}}%
        \whiten{1}\whiten{0}%
        \whiten{-1}\whiten{-2}%
        \llap{\the\SOUL@token}%
     \else
        \underline{\the\SOUL@token}%
     \fi}%
\SOUL@}
\makeatother

% Postaviti datum ispita koji se koristi u ostatku fajla
% prema tom datumu imenovati i glavni .tex fajl!
%
% Taj fajl treba da se zove si1oe_XXX_YY-ZZ.tex , gde je
%   - XXX sifra ispita malim slovima {jan, feb, jun, jul, aug, okt}
%   - YY-ZZ skolska godina, u formatu 2023/2024 |-->  23-24
% Isto ime promeniti i u settings.json !
%
\newcommand{\datumIspita}{xx.xx.xxxx. г}

% Setovati da li se nude opcije za polaganje ispita {true/false}
\setboolean{opcijeZaPolaganje}{true}  


\usepackage[Symbolsmallscale]{upgreek}
\usepackage{fancybox, fancyhdr, graphicx}
\usepackage{enumitem}
\newcommand{\spacer}{\hspace*{5mm}}

\usepackage{amsmath, amsfonts}
\usepackage{icomma}
\usepackage{multirow}
\usepackage[table,xcdraw]{xcolor}

%\usepackage{mathptmx}
\usepackage{newtxmath}
%\usepackage{tempora} % this supports Cyrillic
%\usepackage{newtxtext}

\usepackage{tikz}
\newcommand*\circled[1]{\tikz[baseline=(char.base)]{
            \node[shape=circle,draw,inner sep=1pt] (char) {#1};}}

\let\sigma\upsigma
\let\omega\upomega

\frenchspacing

\newcommand{\unit}[1]{{\rm\,#1}}
\newcommand{\jj}{{\rm j}}
\newcommand{\ee}{{\rm e}}
\newcommand{\de}{{\rm d}}
\newcommand{\uu}{{\rm u}}
\newcommand{\DD}{{\rm D}}

\newcommand{\faz}[1]{\ensuremath{\underline{#1}}}

\newcommand{\ID}[0]{0}

\renewcommand{\infty}{\includegraphics[scale=1]{inf.pdf}}

\pagestyle{fancy}

\fancyhead[C]{ЕТФ у Београду, Катедра за електронику}

\fancyfoot[C]{\footerCenter}
\fancyfoot[RO]{Страна \thepage\ од \pageref{LastPage}}
\fancyfoot[LE]{Страна \thepage\ од \pageref{LastPage}}


\usepackage{tabularx} % in the preamble


\begin{document}

\noindent
\begin{minipage}[t]{0.5\textwidth}
\begin{flushleft}
\vspace*{0pt}
{\large \textbf{Електроника}} \\[2mm]
\ifthenelse{\boolean{ispit}}{Испит}{Колоквијум}: \datumIspita \\
Трајање испита: \trajanjeIspita\ минута \\
Одговорни наставник и асистент: Р. Ђурић и Д. Ђокић
\end{flushleft}
\end{minipage}
\hfill
\begin{minipage}[t]{0.5\textwidth}
\vspace*{0pt}
\ifthenelse{\boolean{ispit}}{
        \begin{tabularx}{\textwidth}{|X|X|X|X||X|X|X|X|||X|} 
        \hline
        1 & 2 & 3 & 4 & 5 & 6 & 7 & 8 & $\sum$ \\
        \hline \hline 
        \begin{minipage}{0.2\textwidth}
        \vspace*{10mm}
        \end{minipage}
        & & & & & & & &   \\
        \hline
        \end{tabularx}
}{
        \begin{tabularx}{\textwidth}{|X|X|X|X|X|} 
        \hline
        1 & 2 & 3 & 4 & $\sum$ \\
        \hline \hline 
        \begin{minipage}{0.2\textwidth}
        \vspace*{10mm}
        \end{minipage}
        & & & &  \\
        \hline
        \end{tabularx}
}
\end{minipage}

\vspace*{2.5mm}

\noindent
\textbf{Напомене}: 
\textit{Испит се полаже интегрално (ради се 3 сата), или преко колоквијума (ради се 2 сата), када се може радити први
колоквијум (задаци 1, 2, 3 и 4), или други колоквијум (задаци 5, 6, 7 и 8). 
Заокружити тачно једну опцију за полагање испита.
Студентима који желе да поправе резултате колоквијума 
при извођењу коначне оцене
у обзир ће се узимати већи број поена између колоквијума 
и дела испита који се односи на
колоквијум. 
} \vspace*{2mm}

\noindent
\textbf{Опција за полагање испита:} 
\hfill
(\textit{i}) I колоквијум, 
\hfill
(\textit{ii}) II колоквијум,
\hfill
(\textit{iii}) Интегрални испит.

\vspace*{5mm}

\noindent
\textbf{Презиме, име и број индекса:}
\hrulefill.

\vspace*{5mm}
\noindent
\boxed{\textbf{1.}}
\begin{minipage}[t]{0.93\textwidth}
\begin{enumerate}
\itemsep0pt
%\begin{itemize}
\item[(a)] \textbf{[0п]}
\end{enumerate}
\end{minipage}

\vspace*{1mm}
%\textbf{\myul{Решење}}:

\newpage


%\vspace*{5mm}
\noindent
\boxed{\textbf{2.}}
\begin{minipage}[t]{0.93\textwidth}
\begin{enumerate}
\itemsep0pt
\item[(а)] \textbf{[0п]}
\end{enumerate}
\end{minipage}



\newpage

\setlist{nolistsep}

%\vspace*{mm}
\noindent
\begin{minipage}[t]{0.8\linewidth}
\noindent
\boxed{\textbf{3.}} 
\vspace*{2mm}
%\setlist{nolistsep}
\end{minipage}
\begin{minipage}[t]{0.2\linewidth}
\begin{flushright}
	\vspace{-10pt}
        %\includegraphics[scale=1]{}
\end{flushright}
\end{minipage}

\vspace*{5mm}
\noindent
%\textbf{Простор за рад:}


\newpage 

%\vspace*{mm}
\noindent
\begin{minipage}[t]{0.7\linewidth}
\noindent
\boxed{\textbf{4.}} \textbf{[0п]} 
\end{minipage}
\begin{minipage}[t]{0.3\linewidth}
\begin{flushright}
	\vspace{-10pt}
        %\includegraphics[scale=1]{fig/t4.pdf}
\end{flushright}
\end{minipage}


\newpage

%\vspace*{5mm}
\noindent
\boxed{\textbf{5.}}
\begin{minipage}[t]{0.93\textwidth}
\begin{enumerate}
\itemsep0pt
%\begin{itemize}
\item[(a)] \textbf{[0п]}
\end{enumerate}
\end{minipage}

\newpage


%\vspace*{5mm}
\noindent
\boxed{\textbf{6.}}
\begin{minipage}[t]{0.93\textwidth}
\begin{enumerate}
\itemsep0pt
%\begin{itemize}
\item[(a)] \textbf{[0п]}

\end{enumerate}
\end{minipage}


\newpage

%\vspace*{mm}
\noindent
\begin{minipage}[t]{0.6\linewidth}
\noindent
\boxed{\textbf{7.}} \textbf{[0п]} 
\end{minipage}
\begin{minipage}[t]{0.4\linewidth}
\begin{flushright}
	\vspace{-10pt}
        %\includegraphics[scale=1]{fig/t7.pdf}
\end{flushright}
\end{minipage}

\vfill

\newpage

%\vspace*{mm}
\noindent
\begin{minipage}[t]{0.7\linewidth}
\noindent
\boxed{\textbf{8.}} \textbf{[0п]}
\end{minipage}
\begin{minipage}[t]{0.3\linewidth}
\begin{flushright}
	\vspace*{0pt}
        %\includegraphics[scale=.833]{fig/t8.pdf}
\end{flushright}
\end{minipage}



\newpage

\textbf{Додатни простор за рад:}

\newpage

\textbf{Додатни простор за рад:}

\end{document}